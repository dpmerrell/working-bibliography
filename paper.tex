\documentclass[14pt]{article}

%\usepackage{showframe}
\usepackage{ragged2e}
\usepackage{hyperref}
\usepackage{amsmath}
%\usepackage{amsfonts}
\usepackage{amssymb}
\usepackage{color}
\usepackage[round]{natbib}


\date{}
\setlength{\topmargin}{0pt}
\setlength{\headheight}{0pt}
\setlength{\headsep}{0pt}
\setlength{\oddsidemargin}{0pt}
\setlength{\evensidemargin}{0pt}
\setlength{\textheight}{9in}
\setlength{\textwidth}{6.50in}
\setlength{\hoffset}{0.0in}
\setlength{\voffset}{-0.0in}
\renewcommand*\rmdefault{ppl}
\setlength{\parindent}{0.0in}
\setlength{\parskip}{14pt plus4pt minus4pt}
%\renewcommand{\baselinestretch}{1.25}


\begin{document}

\title{Working Bibliography}

\author{David Merrell \\ Department of Computer Sciences \\ University of Wisconsin -- Madison}

\maketitle{}

\thispagestyle{empty}


\section{Signaling pathway network inference}

\subsection{Constraint-based approaches}

\paragraph{ \citet{2018-koksal-tps}. Temporal Pathway Synthesizer (TPS). }




\subsection{Dynamic Bayesian Network (DBN) approaches}

\paragraph{ \citet{2017-hill-context}. Context specificity \ldots phosphoprotein profiling.}
The authors build on their previous works 
\citep{2015-spencer-interventional} 
\citep{2014-oates-joint}
to infer context-specific signaling pathways from phosphorylation data.
They consider (4 cell lines) $\times$ (8 perturbagens) $=$ 32 different contexts.
For each context they apply 6 different kinase inhibitor interventions, and incorporate the resulting causal information into their context-specific pathway reconstruction.
\citeauthor{2017-hill-context} validate the novel edges predicted by their method through western blot analysis.

\paragraph{ \citet{2015-spencer-interventional}. Inferring networks from interventional data.}

\paragraph{ \citet{2014-oates-joint}. Joint estimation of multiple related biological networks.}
The authors present a method for inferring multiple related networks simultaneously.
They do so by defining a hierarchical model whose leaves are networks;
they infer the network posteriors via belief propagation.
The results are exact, since the networks' parameters can be marginalized exactly and the hierarchical model contains no loops. 

\paragraph{ \citet{hill-bayesian-2012} Bayesian inference of signaling network topology.}




\subsection{Belief Propagation/Prize-Collecting Steiner Tree approaches}

\paragraph{ \citet{2013-molinelli-perturbation}. Inferring networks for perturbed cells.}
The authors use the belief-propagation method described by \citeauthor{2012-biazzo-steiner} to infer 
MUST REREAD --- DO NOT YET FULLY GRASP THIS ONE.

\paragraph{ \citet{2012-biazzo-steiner}. Performance of a cavity-method-based algorithm for the prize-collecting Steiner tree problem on graphs.}
The authors explain how the PCST objective function can be regarded as a log-probability over trees, and then maximize it (approximately) using a belief-propagation algorithm.
They show that it compares favorably against other PCST solvers---typically based on integer program formulations---on a standard set of PCST benchmarks.
Their method usually converges faster \emph{and} attains higher scores than the others.

\paragraph{ \citet{2011-bailly-bechet-belief}. Finding undetected protein associations in cell signaling by belief propagation.}




\subsection{Other Works}

\paragraph{ \citet{2018-invergo-review}. Review of phosphorylation network reconstruction methods. }

\paragraph{ \citet{2016-hill-community}. Inferring causal molecular networks \ldots community-based effort.}

\paragraph{ \citet{2008-mukherjee-priors}. Network inference using informative priors.}
The authors describe the network inference task and present a family of prior distributions for networks:
$ P(G) \propto \exp\left( \lambda \sum_i w_i f_i(G) \right).$ 
That is, the probability of a graph is some log-linear function of various graph features. 
\citeauthor{2008-mukherjee-priors} use this family of priors in conjunction with 
(a) exact marginal likelihood $P(X | G) = \int P(X | G, \theta) P(\theta) d\theta$, and
(b) MCMC sampling
to estimate expectations of features in the posterior network.
They evaluate their inference method via 
\begin{enumerate}
    \item a simulation study; they estimate the probabilities of edge existence in the posterior network
        and generate an ROC curve for these predictions.
    \item reconstructing a signaling pathway from phosphorylation data; they compare the most likely
        graph against expert knowledge of this particular pathway.
\end{enumerate}

\paragraph{ \citet{2010-vaske-paradigm}. Patient-specific pathway activities from multi-omic data.}

\bibliographystyle{plainnat}
\bibliography{biblio}

\pagebreak

\end{document}
